\documentclass{article}
\usepackage{fullpage}
\usepackage{amsfonts}
\usepackage{amsmath}
\usepackage{amssymb}
\usepackage{tgpagella}
\usepackage[euler-digits]{eulervm}
\usepackage{tikz-cd}
\usepackage{color}
\usepackage{hyperref}
\author{William Schlieper}
\title{Vertex Operator Algebras}
\newcommand{\RR}{\mathbold{R}}
\newcommand{\CC}{\mathbold{C}}
\newcommand{\ZZ}{\mathbold{Z}}
\newcommand{\FF}{\mathbold{F}}
\newcommand{\QQ}{\mathbold{Q}}
\newcommand{\CP}{\mathbold{CP}}
\newcommand{\vac}{|0\rangle}
\newcommand{\Hh}{\mathcal{H}}
\newcommand{\Oo}{\mathcal{O}}
\newcommand{\Kk}{\mathcal{K}}
\newcommand{\Dd}{\mathcal{D}}
\newcommand{\Aa}{\mathcal{A}}
\newcommand{\one}{\mathbold{1}}
\newcommand{\gf}{\mathfrak{g}}
\newcommand{\hh}{\mathfrak{h}}
\newcommand{\ghat}{\widehat{\mathfrak{g}}}
\newcommand{\normord}[1]{:\mathrel{#1}:}
\newcommand{\tk}{\textcolor{red}{*}}
\DeclareMathOperator{\Res}{Res}
\DeclareMathOperator{\End}{End}
\DeclareMathOperator{\Ind}{Ind}
\DeclareMathOperator{\Der}{Der}
\begin{document}
\maketitle
This document is a reconstruction based on my notes of the vertex operator algebra seminar
that ran during Spring Quarter 2016, run by Raphaël Rouquier.  Some notation has been changed (in particular, for the Heisenberg Fock space), some commentary has been added to my handwritten notes, and some definitions and concepts have been moved around, so as to make things more clear for the reader.

\tableofcontents

\section{Introduction (Raphaël Rouquier, March 30)}
\label{sec:intro}

\subsection{Motivation}
\label{sec:mot}
(\tk can probably load this section with references, really)
Monstrous Moonshine may not be one of the most difficult problems in mathematics, but it is perhaps one of the strangest.  The theory involves the observation that, with the exception of the constant term $744$\footnote{which, though it usually isn't, can itself be explained using vertex operator algebras; instead of the Monster, it can be related to a vertex operator algebra obtained from the affine Lie algebra $\widehat{\mathfrak{e}_8}^3$; note that $744=3 \dim \mathfrak{e}_8$.}, the early Fourier terms of the modular $j$-invariant
\[j(\tau)=q^{-1}+744+196884q+21493760q^2+... \quad q=\exp(2 \pi i \tau)\]
correspond to small integer sums of dimensions of representations of the Monster group, the largest sporadic finite simple group: in particular, the smallest nontrivial representation of the Monster is of dimension $196883$, one less than the $q$ term of $j$.  Conway conjectured, and Borcherds proved, the following theorem:
\subsubsection{Theorem (Monstrous Moonshine) (Borcherds (\tk year))}
\label{sec:monmoon}
There is a natural graded representation
\[V^\natural = V_0 \oplus V_1 \oplus ... \]
of the Monster group $M$ such that
\[j(\tau)-744=q^{-1}+\sum_{i=0}^{\infty} \dim V_i q^{i+1}. \]
Furthermore, for $g \in M$, the graded character
\[Tr_{V^\natural}(g;\tau)=q^{-1}+\sum_{i=0}^{\infty} Tr_{V_i}(g) q^{i+1} \]
is a modular function; that is, for some finite index subgroup of $SL_2(\ZZ)$ acting on $\tau$ by Möbius transformations, $Tr_{V^\natural}(g;\tau)$ is left invariant.

\subsection{Vertex Operator Algebras and Conformal Field Theories}
This representation of the monster has the structure of a vertex operator algebra, and, in fact, one good way to \textit{define} the Monster is simply to define it as the automorphisms of $V^\natural$ preserving the vertex operator algebra structure.  Of course, the previous sentence begs the question, quite strongly, of what a vertex operator algebra even \textit{is}.  Here, a hint is provided by the fact that Borcherds's proof uses a variant of the Weyl-Kac character formula to calculate traces, which for affine Kac-Moody algebras (\tk some Kac reference goes here) is linked to modular forms as well.

In fact, vertex operator algebras are related to conformal field theory in $2$ dimensions, where the relevant local symmetries, described in terms of a complex number $z$, cleanly split into the symmetries holomorphic in $z$ and those holomorphic in $\overline{z}$; each of these will correspond to a VOA, and the CFT as a whole will link to a bimodule over both.  For more of this from a physics point of view, there's probably a good physics paper I can reference, and the ``full'' version of this will do so.

\subsection{Formal power series}

Unlike the definitions preferred by physicists, the vertex operator algebra is an algebraic construction based on formal power series, so it is more abstract.  An important space of ``functions'' in this characterization is the \textit{formal distributions} $\CC[[z^{\pm 1}]]$, which consists of all abstract sums
\[a(z)=\sum_{n \in \ZZ} a_n z^n \quad a_n \in \CC. \]
Note that $\CC[[z^{\pm 1}]]$ is \textit{not} a ring; for instance, try to multiply the sum $\sum_{n \in \ZZ} z^n$ with itself.  However, for any Laurent polynomial $P(z) \in \CC[z^{\pm 1}]$ it is possible to multiply $P(z)$ with $a(z)$ and get another formal distribution; furthermore, there exists a natural residue
\[\Res_{z=0} a(z)=\Res_z a(z)=a_{-1} \]
which corresponds to the one defined for functions holomorphic on $\CC^\times$ using a contour integral; the name formal distributions comes from the fact that multiplying and using the residue gives a natural definition of $\CC[[z^{\pm 1}]]$ as a linear dual to the test functions $\CC[z^{\pm 1}]$.

Another way two power series $a(z),b(z) \in \CC[[z^{\pm 1}]]$ can be multiplied is simply to apply them to different variables: the product is simply
\[a(z)b(w)=\sum_{m \in \ZZ}a_mz^m \cdot \sum_{n \in \ZZ}b_nw^n \in \CC[[z^{\pm 1},w^{\pm 1}]], \]
where the formal distributions in two variables are defined in a similar way to those in one.

The immediate question to ask, for any two-variable formal distribution, is: what happens when $z$ approaches $w$?  As is easy to see, this cannot be calculated directly, especially in the cases we are most interested in.  In particular, consider the formal version of the Dirac delta function
\[\delta(z,w)=\sum_{n \in \ZZ} z^n w^{-n-1} \]
which has, much like the Dirac delta function used in analysis, the property that for any Laurent polynomial $P(z) \in \CC[z^{\pm 1}]$, we have
\[P(z) \delta(z,w) = P(w) \delta(z,w). \]
Also like the Dirac delta function, we have:
\begin{align*}
  (z-w) \delta(z,w) &= 0\\
  (z-w)^m \left(\frac{\partial}{\partial z} \right)^n (\delta(z,w))&=0 \quad m>n
\end{align*}
This property of the derivatives of the formal delta function is important, as it implies a value that can be thought of as zero outside of $z=w$; a two-variable formal distribution $a(z,w) \in \CC[[z^{\pm 1},w^{\pm 1}]]$ is called \textit{local} if
\[(z-w)^n a(z,w)=0 \]
for some $n \ge 0$.

\subsection{Vertex algebras}

(\tk not sure if I want to put the actual VA def here or later; probably mention ``creation of boson''?)

The simplest case of a vertex algebra is a commutative one, where the locality condition on the vertex operators is strengthened to actual commutation, or, equivalently, all negative-degree terms are zero.  In this case, we have
\[Y(A,z)=\sum_{n=0}^{\infty} \frac{z^n}{n!}(T^nA) \]
where $T^nA \in V$ corresponds to an endomorphism of $V$ from a commutative algebra structure on $V$.

Non-commutative vertex algebras, however, are already difficult to write down even in the simplest cases and require heavy use of the technique of normal ordering of infinite sums to produce a valid field.

In order to properly relate the purely formal vertex algebra structure to the more geometric structures implied by conformal field theory, one must add extra structure to the vertex algebra.  Let $c \in \CC$; a \textit{vertex operator algebra with central charge $c$} is a vertex algebra with a $\ZZ$-grading (isn't the grading part of the VA def?) and a specific element
\[\omega \in V_2 \quad Y(\omega,z) =: \sum_{n \in \ZZ} L_n z^{-n-2} \]
where the Fourier coefficients $L_n$ satisfy the Virasoro relations
\[[L_m,L_n]=(m-n)L_{m+n}+\frac{c}{12}(m^3-m)\delta_{m,-n} id, \]
the operator $L_{-1}=T$ identifies a lift of $-t^0\partial_t$ with the translation operator, and $L_0|_{V_n}=n \cdot id$ identifies a lift of $-t\partial_t$ with the grading operator.

\subsection{Vertex operator algebras and geometry}
Given a vertex operator algebra, one can connect it to the geometry of algebraic curves in various interrelated ways.  By the mechanisms of algebraic geometry, the fields of a vertex operator algebra $V$ can be thought of as being $\End V$-valued functions on a formal disk; the challenge is to figure out how best to extend these to algebraic curves.

Given a vertex operator algebra $V$, a projective algebraic curve $X$, and a finite collection of representations $M_1,...,M_n$ attached to points $x_1,...,x_2 \in X$, there is a vector space called the space of conformal blocks consisting of functionals for which the process of multiplying an element can be extended from punctured formal neighborhoods of the $x_i$s to the curve $X$ with the points $x_i$ removed.  In fact, using the full Virasoro action, one can extend this from a vector space to the full structure of a (twisted) $\Dd$-module on the moduli space $\mathcal{M}_{g,n}$ of genus $g$ algebraic curves with $n$ points.

In particular, given $X=\CP^1$, points $0,1,\infty$, and $V$ itself attached at $0$ and $1$ and the restricted dual $V^\wedge$ attached at $\infty$, the vertex operators themselves turn out to be conformal blocks; if more points are attached, then the corresponding conformal blocks will turn out to be the composed vertex operators, which can be combined into an operadic approach.

(\tk think about how to summarize factorization and chiral algebras (Beilinson-Drinfeld) when it's not 10:30 pm)

(\tk probably mention rational VOAs here?  I think I might wait 'til I see what Kevin has planned)

\subsection{Examples}
\begin{itemize}
\item Given an affine Lie algebra and a level $k$, one can construct a vertex algebra whose representations are representations of the affine Lie algebra at level $k$; when $k \ne - h^\wedge$, this is a vertex operator algebra.  When $k$ is also a positive(?) integer, then 
\item There is a construction that, given an even integral lattice, produces a rational vertex operator algebra.  In this case, the graded character of a representation can be expressed naturally in terms of theta functions.  If the lattice is a root lattice of type $ADE$, then the corresponding lattice VOA is isomorphic to the rational vertex operator algebra derived from the corresponding level $1$ affine Lie algebra VOA.
\item When the lattice used in the lattice construction is the Leech lattice, there is an involution acting as $-1$ on level $1$ elements.  When one takes the subset of the lattice VOA preserved by that involution, this itself is a VOA; taking the direct sum of this VOA with another representation of it acts as a twisted VOA which produces the Monster vertex operator algebra $V^\natural$.
\end{itemize}

\section{Vertex algebras: definition and Heisenberg (Jeremy Brightbill, April 13)}
\label{sec:defheis}

\subsection{Formal distributions}
(\tk is this section redundant with the intro section?)
Let $R$ be a $\CC$-algebra.  Then, for any integer $n$, the \textit{$R$-module of formal distributions} $R[[z_1^{\pm 1},...,z_n^{\pm 1}]]$ can be defined simply as the set of $\ZZ^n$-tuples of elements of $R$, thought of accordingly as formal power series extending in all directions.  Formal distributions generally do not form a ring; however, given two formal distributions $a(z),b(z) \in R[[z]]$ defined as
\begin{align*}
  a(z)&=\sum_{n \in \ZZ}a_nz^n\\
  b(z)&=\sum_{n \in \ZZ}b_nz^n
\end{align*}
there is a product in \textit{different} variables
\[a(z)b(w) = \sum_{m,n \in \ZZ}a_mb_nz^mw^n. \]
Furthermore, the $R$-module structure can be extended to $R[z_1^{\pm 1},...,z_n^{\pm 1}]$ by multiplying and interleaving the finite with the infinite sum.

(\tk mention the formal delta function again?)

To ``evaluate'' formal distributions, since getting any actual values in the context of infinite power series is impossible, the correct idea is instead to use the residue as the formal version of the integral around a small circle around $0$: as such, 
\[\Res_{z=0}a(z)=a_{-1} \]
and, in fact, for vertex algebra purposes it makes more sense to number formal distributions using the residue, which we will denote as
\begin{align*}
  a(z)&=\sum_{n \in \ZZ}a_{(n)}z^{-n-1}\\
  a_{(n)}&=\Res_{z=0}
\end{align*}

\subsection{Fields and locality}
Let $V$ be a vector space over $\CC$.  We define a \textit{field} to be an element
\[A(z) \in \End_\CC(V)[[z^{\pm 1}]] \quad A(z)=\sum_{n \in \ZZ} A_{(n)}z^{-n-1} \]
of formal distributions over the endomorphisms of $V$, such that, for all $v \in V$ there exists an $N$ such that $A_{(n)}v=0$ for all $n>N$; that is, 
\[A(z)v \in V((z)) \quad \forall v \in V. \]
Note that $A$ being a field is a weaker condition than $A$ being a formal Laurent series over $\End(V)$; the latter would include a global $N$ for all $v \in V$.  Furthermore, to work over $\CC$, we can consider a pair of $\varphi \in V^*, v \in V$, and note that the evaluation
\[\langle \varphi, A(z)v \rangle \in \CC((z)) \]
is itself a formal Laurent series.

If $V$ is also $\ZZ$-graded, then a field $A(z)$ is said to have \textit{conformal dimension} $N \in \ZZ$ if $A_{(n)}$ is homogeneous of degree $-n+N-1$; that is, if $z$ is thought of as having degree $-1$, then a conformal dimension $N$ field will be homogeneous of degree $N$.

Given two fields $A(z),B(w)$, we would like to be able to compare $A(z)B(w)$ and $B(w)A(z)$.  However, for them to be equal as formal distributions over $\End V$ is a very strong condition.  Given $v \in V, \varphi \in V^*$, we'd like to compare the two expressions
\begin{align}
  \langle \varphi, A(z)B(w) v \rangle &\in \CC((z))((w)) \label{expansion1}\\
  \langle \varphi, B(w)A(z) v \rangle &\in \CC((w))((z)). \label{expansion2}
\end{align}
(might be able to add some more about why they lie in the given formal Laurent series)
For these to be equal as formal distributions in two variables, they must both be expressible as elements of $\CC[[z,w]][z^{-1},w^{-1}]$, which is a very strong constraint on the possible coefficients of the two fields for negative powers of $z$ and $w$.  For example, consider the function $1/(z-w)$.  It can be expanded into formal power series in two ways, depending on whether it is expanded in $z$ or $w$:
\begin{align}
  \frac{1}{z-w}&=\sum_{n<0}z^nw^{-n-1} \in \CC((z))((w))\\
  \frac{1}{z-w}&=-\sum_{n \ge 0}z^nw^{-n-1} \in \CC((w))((z)).
\end{align}
Notice that the difference between the two power series is precisely the formal delta function mentioned before.  Indeed, the diagram
\[
  \begin{tikzcd}
    \CC[[z,w]][z^{-1},w^{-1},(z-w)^{-1}] \arrow[r,hook,"\textrm{expand at }w"] \arrow[d,hook,"\textrm{expand at }z"] & \CC((z))((w)) \arrow[d,hook,"\textrm{inclusion}"]\\
    \CC((w))((z))\arrow[r,hook,"\textrm{inclusion}"]& \CC[[z^{\pm 1},w^{\pm 1}]]
  \end{tikzcd}
\]
does \textit{not} commute!  However, the difference between the two maps will always be local in the sense that multiplyng some power of $(z-w)$ will take it to zero, thus providing an alternate characterization of locality.
\subsubsection{Definition (locality)}
Let $A(z),B(z) \in \End (V)[[z^{\pm 1}]]$ be fields over a vector space $V$.  We say that $A(z)$ and $B(w)$ are local with respect to each other if, for all $v \in V, \varphi \in V*$ there exists
\[f_{v,\varphi} \in \CC[[z,w]][z^{-1},w^{-1},(z-w)^{-1}] \]
such that the expressions (\ref{expansion1}) and (\ref{expansion2}) are both expansions of $f_{v,\varphi}$, and that the order of the pole of $(z-w)$ in $f_{v,\varphi}$ is uniformly bounded. (The uniform bound doesn't seem true of $Y(A,z)Y(B,w)=Y(Y(A,z-w)B,w)$ but Ben-Zvi,Frenkel says it's true.  Am I confused?)

Note that locality is not in fact a trivial condition even in the case $A=B$.

\subsubsection{Proposition}
$A(z)$ and $B(w)$ are local if and only if there exists an $N$ such that $(z-w)^N[A(z),B(w)]=0$ as elements of $\End(V)[[z^{\pm 1},w^{\pm 1}]]$.

\subsection{Vertex Algebras}

\subsubsection{Definition}
A \textit{vertex algebra} consists of the following data:
\begin{itemize}
\item A vector space $V$ over $\CC$.
\item A distinguished element $\vac \in V$, called the \textit{vacuum vector}.
\item A linear map $T: V \rightarrow V$ called \textit{translation}.
\item A linear map $Y(-,z): V \rightarrow \End(V)[[z^{\pm 1}]]$ such that for any $A \in V$, the formal distribution
  \[Y(A,z)=\sum_{n \in \ZZ} A_{(n)} \]
  is a field.  These are called the \textit{vertex operators}.
\end{itemize}
satisfying the following properties:
\begin{itemize}
\item The vacuum axiom:
  \begin{align}
    Y(\vac,z)&=id_Vz^0\\
    Y(A,z)\vac &\in V[[z]],\\
    Y(A,z)\vac|_{z=0}&=A\quad \forall A \in V.
  \end{align}
\item The translation axiom:
  \begin{align}
    T \vac &= 0\\
    [T,Y(A,z)]&=\partial_z Y(A,z)
  \end{align}
\item The locality axiom: For all $A,B \in V$, the fields $Y(A,z)$ and $Y(B,w)$ are local to one another.
\end{itemize}

\subsubsection{Graded vertex algebras}
A vertex algebra $V$ is ($\ZZ-$)graded if:
\begin{itemize}
\item $V$ is $\ZZ$-graded; i.e.
  \[V = \bigoplus_{n \in \ZZ} V_n \]
\item $\vac \in V_0$
\item $T$ is homogeneous of degree $1$; that is, $TV_n \subseteq V_{n+1}$
\item $Y(A,z)$ has conformal dimension $m$ for all $A \in V_m$
\end{itemize}
Most of the vertex algebras seen in the wild are graded; in particular, a vertex operator algebra is by definition a graded vertex algebra.

\subsubsection{Vertex algebra homomorphisms}
A homomorphism
\[\rho: (V,\vac,T,Y(-,z)) \rightarrow (V',\vac',T',Y'(-,z)) \]
is a linear map $\rho: V \rightarrow V'$ such that:
\begin{align}
  \rho \circ T &= T'\circ \rho\\
  \rho(Y(A,z)B)&=Y'(\rho(A),z)\rho(B) \quad \forall A,B \in V
\end{align}
This definition of a vertex algebra morphism also lets you define, straightforwardly, the idea of a subalgebra, that of a quotient, and that of an ideal as the kernel of a quotient.

Furthermore, one can define the tensor product of two vertex algebras such that $V \rightarrow V \otimes W$ and $W \rightarrow V \otimes W$ are vertex algebra homomorphisms.

\subsubsection{Properties of vertex algebras}
By the translation and vacuum axioms, we have
\begin{align}
  A_{(-2)}\vac &= \partial_z Y(A,z) \vac |_{z=0}\\
  &=[T,Y(A,z)]\vac|_{z=0}\\
  &=TA.
\end{align}
Similarly, $T^nA=n!A_{(-n-1)}\vac$.  Equivalently,
\[Y(A,z) \vac = \sum_{n=0}^\infty \frac{1}{n!} T^nA z^n = e^{zT}A \]
which can be interpreted as relating $T$ to translation, as interpreting $\exp$ of operators in the same way with differentials gives the relation $e^{w \partial_z}f(z)=f(z+w)$ for $f$ a polynomial.

In general, if $Y(A,z)=A_{(-1)}z^0$ for all $A \in V$, then $T=0$, $Y: V \rightarrow \End(V)$ will simply be the map taking an element to left-multiplication by it, locality implies commutativity, and $V$ picks up the structure of a commutative algebra with unit.

If $Y(A,z)$ and $Y(B,w)$ commute for all $A,B \in V$, then all vertex operators will lie in $\End(V)[[z]]$ and multiplication can be defined as
\[A \circ B = Y(A,z)B|_{z=0}. \]
Using this multiplication, $V$ is a commutative algebra, with $T$ acting as a derivation, and
\[Y(A,z)B=(e^{zT}A) \circ B \quad \forall A,B \in V \]

(\tk It's probably worth it to mention the associativity/Cousin property $Y(Y(A,z-w)B,w)"="Y(A,z)Y(B,w)$ here? and $Y(TA,z)$? Both come from the state-field correspondence, which we didn't go over)

\subsection{The Heisenberg vertex algebra}

\subsubsection{The Heisenberg Lie algebra}


Start by considering the algebra of formal Laurent series $\CC((t))$ as a vector space, and then that vector space as a Lie algebra with trivial bracket.  Then, the Heisenberg Lie algebra is defined as the central extension of $C((t))$ with the short exact sequence
\[1 \rightarrow \CC \cdot \one \rightarrow \Hh \rightarrow \CC((t)) \rightarrow 0 \]
such that (after choosing arbitrarily a vector space splitting of $H$)
\[[f(z),g(z)]=-\Res(f\,dg) \cdot \one. \]
(Note that the bolded number one, $\one$, is a different character from the usual number one, $1$.)  

Alternatively, if we extended instead the subalgebra $\CC[t^{\pm 1}]$ of $\CC((t))$ consisting of Laurent polynomials, we would get the subalgebra $\Hh' \subsetneq \Hh$.  A basis for $\Hh'$ which serves as a topological basis for $\Hh$ is as follows: define $b_n=t^n$, and then the set $\{b_n\}_{n \in \ZZ} \cup \{\one\}$ form the basis.  The Heisenberg algebra can also be defined using explicit generators and relations:
\begin{align}
  [b_n,\one]&=0\\
  [b_m,b_n]&=-\Res(t^n\,dt^m)\cdot\one\\
  &=-\Res(mt^{n+m-1}dt)\cdot\one\\
  &=n\delta_{m,-n}\cdot\one
\end{align}

(This is a part that I'm writing a bit differently so as to not have to use a bunch of tildes)
Note that it is possible to give the Heisenberg algebra a ``triangular decomposition''(\tk is this the right name?) as 
\[\Hh \simeq_{\CC-Mod} t^{-1}\CC[t^{-1}] \oplus \one \oplus \CC[[t]].  \]

\subsubsection{The Fock space}
\label{sec:fockspace}

Let $\CC_0$ be the one-dimensional representation of $\CC \cdot\one \oplus \CC[[t]]$ that is acted on by $\CC[[t]]$ trivially and by $\one$ by $1$.  Then, we define the \textit{Heisenberg vertex algebra} or \textit{Fock space} as
\[\pi:=\Ind_{\CC \cdot \one \oplus \CC[[t]]}^\Hh\CC_0=U(\Hh) \otimes_{U(\CC \cdot \one \oplus \CC[[t]]}\CC_0. \]
Recall that, given a Lie algebra $\gf$, the universal enveloping algebra $U(\gf)$ is the quotient of the free algebra generated by $\gf$ by the two-sided ideal generated by $gh-hg-[g,h]$ for all $g,h \in \gf$; in particular, for an abelian Lie algebra, it is isomorphic to the polynomial algebra generated by $\gf$.

By the Poincaré-Birkhoff-Witt theorem, there is a canonical basis for $\pi$ as follows: Let $\vac$ be the vector $1 \otimes 1$ killed by $\CC[[t]]$.  Then, the Heisenberg vertex algebra has the following isomorphism: 
\[\pi = U(t^{-1}\CC[t^{-1}])\cdot \vac \simeq U(t^{-1}\CC[t^{-1}])=\CC[b_{-1},b_{-2},...]  \]
where the isomorphism is found simply by applying the relevant element to $\vac$.  Note that, under this isomorphism, the actions of the various basis elements of the Heisenberg Lie algebra are as follows:
\begin{align}
  b_n, n<0&: n\frac{\partial}{\partial b_{-n}}\\
  b_0&:0\\
  b_{n},n>0&:b_n \cdot\\
  \one&:id.
\end{align}
The vector space $\pi$ can be given the structure of a graded vertex algebra as follows:
\begin{itemize}
\item The vacuum vector is, as already implied, the canonical vector $\vac$ killed by $\CC[[t]]$.
\item Translation can be defined as follows:
  \begin{align*}
    T\vac&=0\\
    [T,b_j]&=-jb_{j-1}\quad j<0
  \end{align*}
  Given these two relations, an induction argument lets one find the translation operator for any $v \in \pi$.
\item The grading also follows from the grading on the Lie algebra making $b_j$ homogeneous degree $-j$, as well as $\vac$ necessarily being degree $0$.
\item The vertex operator $Y(\vac,z)=id_\pi$ is known by the definition.  In all other cases, it will be possible to reconstruct all the vertex operators from that and the simplest vertex operator
\[Y(b_{-1}\vac,z)=\sum_{n \in \ZZ}b_nz^{-n-1}=:b(z).\]
\end{itemize}
For the elements $b_{-k} \vac$, the translation axiom shows
\[b_{-k} \vac = \frac{1}{(k-1)!}T^{k-1}b_{-1}\vac \]
and so we can calculate
\[Y(b_{-k}\vac,z)=\frac{1}{(k-1)!}\left(\frac{\partial^{k-1}}{\partial z^{k-1}} b(z) \right) \]
To find the vertex operators for states with more than one $b_j$ involved, a more complicated approach is necessary.  Naively, one might expect to find something like ``$Y(b_{-1}^2\vac,z)=b(z)^2$'', but there is a problem: the Fourier coefficients of the naively computed $b(z)^2$,
\[\sum_{k,l \in \ZZ; k+l=n}b_kb_l=("b(z)^2")_{(n+1)}, \]
do not exist.

In the case $n \ne 0$, however, these terms can be finessed into existing.  Note that $b_k$ and $b_l$ will commute in this case, as $k+l \ne 0$.  As such, for all but finitely many $(k,l)$ pairs, one of the two terms will correspond to a positive number; for all $A \in \pi$, all but finitely many positive terms will act as $0$ on $A$, so by rearranging the terms, we can see that
\[\sum_{k+l=n}b_kb_lA \]
can be given a definite value.

For $n=0$, however, this runs into a problem, as $b_k$ and $b_{-k}$ do not commute.  Attempting this calculation ends up as follows:
\begin{align*}
  \sum_{n \in \ZZ}b_nb_{-n}A&=\sum_{n < 0}b_nb_{-n}A + \sum_{n \ge 0}b_nb_{-n}A\\
  &=\sum_{n<0}b_nb_{-n}A+\sum_{n \ge 0}b_{-n}b_nA+nA
\end{align*}
which involves the divergent sum $\sum_{n=0}^\infty n$.  This is clearly a problem, especially in the computation of formal power series.

\subsubsection{Normal ordering}

The solution, called \textit{normal ordering}, is just to rearrange the terms such that positive-indexed terms appear to the right of negative-indexed terms anyway, ignoring commutation.\footnote{Those of you familiar with the Riemann zeta function, or with a certain internet controversy from 2014, are probably wondering where the factor of $1/12$ went.  Such people should attempt to prove directly that $\frac{1}{2}\normord{b(z)b(z)}$ is local with itself.}  Let $A(z),B(w)$ be fields over some variables.  Then, define the \textit{normal-ordered product} of the terms as follows:
\[\normord{A(z)B(w)} = \sum_{n \in \ZZ}\sum_{m<0}A_{(m)}B_{(n)}z^{-m-1}w^{-n-1}+\sum_{n \in \ZZ}\sum_{m \ge 0}B_{(n)}A_{(m)}z^{-m-1}w^{-n-1}. \]
To normally-order a product of more than two fields, use the right-associative expansion; that is,
\[ \normord{A(z)B(w)C(u)}  =\normord{ A(z) (\normord{B(w)C(u)} )} \]

With this definition of normal ordering, we can now give the formula for the vertex operator:

Let $k$ be a positive integer and $j_1,...,j_k$ be integers $\le -1$.  Then,
\[Y(b_{j_1}...b_{j_k}\vac,z)=\frac{1}{\prod_{i=1}^k(-j_i-1)!} \normord{\frac{\partial^{-j_1-1}}{\partial z^{-j_1-1}}(b(z))...\frac{\partial^{-j_k-1}}{\partial z^{-j_k-1}}b(z)} \]

\subsubsection{Proof of $\pi$'s vertex algebra structure}
Now, we wish to show that $\pi$ is indeed a vertex algebra.

First, the vacuum axiom.  That $Y(\vac,z)=id_\pi z^0$ is true by definition.  For $A=b_{-k}\vac$, we have
\begin{align}
  Y(b_{-k}\vac,z)&=\left(\frac{1}{(k-1)!} \frac{\partial^{k-1}}{\partial z^{k-1}} b(z) \right) \vac\\
  &=\sum_{n \in \ZZ}\frac{(-n-1)(-n-2)...(-n-k)}{(k-1)!} b_n z^{-n-k} \vac\\
  &=\sum_{n=0}^\infty \binom{n+k-1}{k-1}b_{-n-k}\vac z^n \in \pi[[z]]\\
  Y(b_{-k}\vac,z)|_{z=0}&=b_{-k}.
\end{align}
For more general $A$, we induct on the word length.  Let $A \in \pi$ be such that the vacuum axiom is known for $A$.  Then, we have
\begin{align}
  Y(b_{-k}A,z)&= \sum_{m \in \ZZ}(b_{-k}A)_{(m)}z^{-m-1}\\
  (b_{-k}A)_{(m)}&=\frac{1}{(k-1)!}\sum_{n \le -k}(-n-1)...(-n-k+1)b_nA_{(m-n-k)}\\
  &+\sum_{n \ge 0}(-n-1)...(-n-k+1)A_{(m-n-k)}b_n
\end{align}
For $n \ge 0$, every $b_n$ annihilates $\vac$.  For $n \le -k$, $n+k \ge 0$ so $A_{(m-n-k)}$ annihilates $\vac$ for $m \ge 0$, so that $Y(b_{-k}A,z)\vac \in \pi[[z]]$.

For $m=-1$ (that is, the value at $0$), all values of $n$ other than $n=-k$ will contain a term annihilating $\vac$, so
\begin{align}
  Y(b_{-k}A,z)\vac|_{z=0}&=\frac{(k-1)!}{(k-1)!}b_{-k}A_{(-1)}\vac\\
  &=b_{-k}A.
\end{align}

\tk My notes say the translation axiom is true by definition, but I think there's a few minor things that need to be checked with normal ordering.

For locality, we again use an induction, but a more complicated one.  First, we show that $b(z)$ is local to itself; that all $Y(b_{-k}\vac,z)$ are mutually local will follow from applying derivatives.  The induction step relies on repeated application of a key lemma called Dong's Lemma.

The self-locality for $b(z)$ follows from
\begin{align}
  [b(z),b(w)]&=\sum_{m,n \in \ZZ}[b_n,b_m]z^{-n-1}w^{-m-1}\\
  &=\sum_{n \in \ZZ}[b_n,b_{-n}]z^{-n-1}w^{n-1}\\
  &=\sum_{n \in \ZZ}nz^{-n-1}w^{n-1}\\
  &=\frac{\partial}{\partial w}\delta(z,w)=-\frac{\partial}{\partial z}\delta(z,w),
\end{align}
which satisfies
\[(z-w)^2 \frac{\partial}{\partial w}\delta(z,w)=0.\]  
For other $Y(b_{-k}\vac,z)$, simply note that the multiple derivatives will be killed by higher powers of $z-w$.  The induction step follows from the following lemma:

\subsubsection{Dong's Lemma}
Suppose that $A(z),B(w),C(u)$ are mutually local fields.  Then, $\normord{A(w)B(w)}$ and $C(u)$ are local to each other.
\subsubsection{Reconstruction theorem}
\label{sec:recthm}
This process of generating the vertex operators from the single vertex operator $b(z)$ can be generalized to a more general theorem building up a vertex operator algebra from a distinguished set of vertex operators.

Let $V$ be a vector space, and fix the following:
\begin{itemize}
\item An element $\vac \in V$, to be the vacuum vector.
\item An endomorphism $T \in \End(V)$, to be the translation operator.
\item A countable totally ordered set $S$
\item A set $\{a^s\}_{s \in S}$ of vectors in $V$.
\item A set $a^s(z)=\sum_{n \in \ZZ}a^s_{(n)}z^{-n-1}$ of fields over $V$.
\end{itemize}
Suppose:
\begin{enumerate}
\item For all $s \in S$, $a^s(z)\vac=0$ and $a^s(z)\vac|_{z=0}=a^s$, hardwiring the vacuum axiom for all $a^s$.
\item $T\vac=0$ and $[T,a^s(z)]=\partial_za^s(z)$, hardwiring the translation axiom.
\item All $a^s(z)$ are mutually local.
\item $V$ has a Poincaré-Birkhoff-Witt-style basis consisting of elements
\[
  \{a^{s_1}_{(j_1)}...a^{s_m}_{(j_m)}\vac\}\quad
  j_1 \le ... \le j_m\quad
  s_i \le s_{i+1}\textrm{ if }j_i=j_{i+1}.
\]
\end{enumerate}
Then, there is a vertex algebra structure on $V$ given by the equation
\[
  Y(a^{s_1}_{(j_1)}...a^{s_m}_{(j_m)}\vac,z)=\frac{1}{\prod_{r=1}^m(j_r-1)!} \normord{\prod_{r=1}^m \frac{\partial^{-j_r-1}}{\partial z^{j_r-1}}a^{s_r}(z)}.
\]
(\tk I should figure out how to make the $:$s bigger)

\section{Lattice vertex algebras and affine algebras (Dustan Levenstein, April 20)}
\label{sec:latticeaffine}

\subsection{Affine Kac-Moody algebras}
Let $\gf$ be a finite-dimensional simple Lie algebra over $\CC$.  Define the \textit{loop algebra} as
\begin{align*}
  L\gf=\gf((t))&=\gf \otimes \CC((t))\\
  [A \otimes f(t),B \otimes g(t)]&=[A,B] \otimes f(t)g(t)
\end{align*}
The \textit{affine Lie algebra} or \textit{affine Kac-Moody algebra}(\tk doesn't the latter also have a derivation?) $\ghat$ is defined as a central extension of $L\gf$
\[0 \rightarrow \CC \cdot K \rightarrow \ghat \rightarrow L\gf \rightarrow 0 \]
as follows: first, we need an invariant form on $\gf$.  Start with the Killing form
\[(x,y)_K=Tr_\gf(ad_xad_y) \]
and normalize using the \textit{dual Coxeter number} $h^\vee$
\[(x,y)=\frac{1}{2h^\vee}(x,y)_K; \]
this invariant form has the property that if $v$ is a maximum-length root vector, $(v,v)=2$; also, if $\gf=\mathfrak{sl}_n$, then $(x,y)=Tr_{\CC^n}(xy)$ for $\CC^n$ the natural representation.

To define $\ghat$, use the natural vector space splitting of $L\gf$ and deform the Lie bracket to
\[[A \otimes f(t),B \otimes g(t) = [A,B] \otimes f(t)g(t) - \Res_{t=0}(f\,dg) (A,B)K. \]

\subsection{The vacuum representation of $\ghat$}
Much like the Heisenberg vertex algebra, observe the decomposition
\[\ghat = t^{-1}\gf[t^{-1}] \oplus \CC \cdot K \oplus \gf[[t]] \]
where $\gf[[t]]$ is defined in analogy to $\gf((t))$.  Let $k \in \CC$; define $\CC_k$ to be the representation of $\gf[[t]] \oplus \CC \cdot K$ on which $\gf[[t]]$ acts by $0$ and $K$ acts by $k$.  Similarly to the construction of the Heisenberg vertex algebra in (\ref{sec:fockspace}), we define the \textit{vacuum representation at level $k$} as follows:
\[V_k(\gf):=\Ind_{\CC \cdot K \oplus \gf[[t]]}^{\ghat}\CC_k = U(\ghat) \otimes_{U(\CC \cdot K \oplus \gf[[t]])}\CC_k. \]
Again, let $\vac$ be the canonical vector $1 \otimes 1$ killed by $\gf[[t]]$; to define a basis for $V_k(\gf)$, we first let $d=\dim \gf$, let $\{J^1,...,J^d\}$ be a basis for $\gf$, define the topological basis
\[\{J^a_n=J^a \otimes t^n|1 \le a \le d, n \in \ZZ \} \]
for $\ghat$.  With this, we can use the Poincaré-Birkhoff-Witt theorem to find a basis
\[J^{a_1}_{n_1}...J^{a_m}_{n_m}\vac \quad n_1 \le ... \le n_m < 0 \quad a_i \le a_{i+1} \textrm{ when }n_i=n_{i+1} \]
for $V_k(\gf)$.

Give $V_k(\gf)$ a vertex algebra structure as follows:
\begin{itemize}
\item Give $V_k(\gf)$ a grading such that $\deg \vac=0$ and $J^a_n$ has degree $n$,
\item $\vac$, as expected, is the vacuum vector,
\item $T\vac=0, [T,J^a_n]=-nJ^a_{n-1}$ will be sufficient to define $T$,
\item $Y(\vac,z)=id$,
\item \[Y(J^a_{-1}\vac,z)=J^a(z):=\sum_{n \in \ZZ}J^a_nz^{-n-1}, \]
\item and note that the conditions for the reconstruction theorem in (\ref{sec:recthm}) hold, giving the structure.
\end{itemize}
Note that $J^a(z)$ can be thought of as a pullback of the operator $\sum_{n \in \ZZ}(J^a \otimes t^n)z^{-n-1}$ which ``looks like'' $J^a\delta(t,z)$, making it an ``operator-valued $\delta$ function''.
\subsection{The Virasoro algebra}
\label{sec:virasoro}
Let $\Oo=\CC[[t]]$ and $\Kk=\CC((t))$.  Define the Lie algebra of derivations of $\Kk$ as $\Der \Kk = \CC((t)) \partial_t$, and define the \textit{Virasoro Lie algebra} as the central extension
\[0 \rightarrow \CC \cdot C \rightarrow Vir \rightarrow \Der \Kk \rightarrow 0 \]
such that
\[[f(t)\partial_t,g(t)\partial_t]=(fg'-f'g)\partial_t+\frac{1}{12}(\Res_{t=0}fg'''\,dt)C. \]

There exists a topological basis of $Vir$ by $C$ and $L_n=-t^{n+1}\partial_t$ for $n \in \ZZ$, with commutation relations
\[[L_n,L_m]=(n-m)L_{n+m}+\frac{n^3-n}{12} \delta_{n,-m}C. \]

Note that this definition depends on the coordinates; there exists a coordinate-free version of the Virasoro algebra, as $H^2(\Der \Kk)=\CC$ so it is essentially unique (aside from the constant $1/12$), but there does not exist a coordinate-free \textit{splitting} of the algebra.

Define
\[T(z) := \sum_{n \in \ZZ}L_nz^{-n-2}; \]
then the Virasoro commutation relations can be summed up as
\[[T(z),T(w)]=\frac{C}{12}\partial^3_w \delta(z,w)+zT(w)\partial_w\delta(z-w)+\partial_wT(w)\delta(z-w). \]
In particular, $T(z)$ is local to itself.

Let $c \in \CC$. To define the \textit{Virasoro vertex algebra} $Vir_c$, start with the subalgebra $\Der \Oo=\CC[[t]]\partial_t$ and the representation $\CC_c$ on which $\Der \Oo$ acts as $0$ and $C$ acts by $c$; we then have
\[Vir_c=\Ind_{\Der \Oo \oplus \CC \cdot C}^{Vir}\CC_c. \]
We call $c$ the \textit{central charge} and say that $Vir_c$ has central charge $c$.

The vertex algebra structure is as follows:
\begin{itemize}
\item $\vac$ is simply the vector $1 \otimes 1$ in the tensor definition of $\Ind$.
\item $Vir_c$ has a Poincaré-Birkhoff-Witt basis
\[L_{j_1}...L_{j_m}\vac \quad j_1\le ... \le j_m < -1. \]
\item $\deg \vac=0, \deg L_n=-n$ is sufficient to give a grading on the algebra
\item $T=L_{-1}$
\item \[Y(L_{-2}\vac,z)=T(z)=\sum_{n \in \ZZ}L_nz^{-n-2}. \]
\item These are the necessary conditions for the reconstruction theorem in \ref{sec:recthm} for $S$ the singleton defining $L_{-2}\vac$ to determine the vertex algebra structure on $Vir_c$.
\end{itemize}

\subsection{Vertex operator algebras}
A graded vertex algebra $V$ is called \textit{conformal} or a \textit{vertex operator algebra} of central charge $c$ if we have a vector $\omega \in V_2$ such that the operators $L^V_n$ defined by
\[Y(\omega,z)=\sum_{n \in \ZZ}L^V_n z^{-n-2} \]
satisfy the Virasoro relations with central charge $c$; furthermore, $L^V_{-1}=T$, and $L^V_{0}|_{V_n}=n\cdot id$, relating $L_0$ to the grading.

The vector $\omega$ is called a \textit{conformal vector}.

Examples of vertex operator algebras include:
\begin{itemize}
\item $Vir_c$ itself, with $\omega=L_{-2}$.  In fact, any vertex operator algebra $V$ of central charge $c$ comes equipped with a map $f:Vir_c \rightarrow V$ such that $\omega=f(L_{-2}\vac)$, and vertex algebras equipped with such a map are precisely the vertex operator algebras.
\item The Heisenberg Fock space $\pi$ has a one-parameter family of conformal vectors: given $\lambda \in \CC$,
\[\omega_\lambda=\left(\frac{1}{2}b_{-1}^2+\lambda b_{-2}\right)\vac \]
is a conformal vector with central charge $c=1-12\lambda^2$.
\item $V_k(\gf)$ has a conformal vector in the case where $k \ne -h^\vee$, using what is called the Sugawara construction.

Given $d=\dim \gf$, pick a basis $J^1,...,J^d$ of $\gf$, and let $J_1,...,J_d$ be the dual basis of $\gf$ defined such that $(J^i,J_j)=\delta_{i,j}$ for the bilinear form defined earlier.  Define the \textit{Sugawara operator} as
\[S:=\frac{1}{2} \sum_{a=1}^d J^a_{-1}(J_a)_{-1} \vac, \]
which is independent of the basis chosen.  Then, $(1/(k+h^\vee))S$ is a conformal vector with conformal charge
\[c=\frac{k \dim \gf}{k+h^\vee}. \]
(At the \textit{critical level} $k=-h^\vee$, the operators $S_{(j)}$ mutually commute instead of forming the Virasoro algebra.)
\end{itemize}

\subsection{Vertex operator algebras associated to integral lattices}

\subsubsection{Heisenberg representations}

Recall $\Hh$ and $\Hh'$, the Heisenberg algebras, with (topological) basis given by $\one$ and $b_n \quad n \in \ZZ$, with
\begin{itemize}
\item $\one$ central
\item $[b_m,b_n]=m\delta_{m,-n}\one$.
\end{itemize}
Define the Heisenberg Weyl algebra as
\[\widetilde{\Hh}=U(\Hh')/(1-\one); \]
note that its representations are the representations of $\Hh'$ with $\one$ acting as $1$.

Fix a scalar $\lambda \in \CC$.  Define $\pi_\lambda$ to be the $\widetilde{\Hh}$-module generated by an element $|\lambda\rangle$ with relations
\begin{align*}
  b_n|\lambda\rangle&=0 \quad \forall n>0\\
  b_0|\lambda\rangle&=\lambda|\lambda\rangle;
\end{align*}
that is, let $\pi_\lambda$ be an induction of the one-dimensional $U(\CC \cdot \one \oplus \CC[[t]])$-module on which $b_0$ acts as $\lambda$.  Much like for $\pi=\pi_0$, we have a Poincaré-Birkhoff-Witt basis
\[\pi_\lambda=\CC[b_n]_{n<0}|\lambda\rangle \]
with the action of $b_i$ related to the Weyl algebra action in much the same way as $\pi_0$'s are.

Note that up to isomorphism, the irreducible graded $\widetilde{\Hh}$-modules with gradation (such that $\deg b_n=-n$) bounded below are precisely the modules $\pi_\lambda$.

In a similar process\footnote{Actually the same process, once representations are defined in (FORWARD REFERENCE)} to the vertex operator of the conformal vector $\omega=\frac{1}{2}b_{-1}^2\vac$ on $\pi$, there exists an action of $Vir$ on $\pi_\lambda$ given by
\[L_n = \frac{1}{2}\sum_{m \in \ZZ}:b_mb_{n-m}: \quad n \in \ZZ \]
with central charge $1$; in particular, $\deg(|\lambda\rangle)|\lambda\rangle=L_0|\lambda\rangle=\frac{1}{2}\lambda^2|\lambda\rangle$ is a canonical degree for the generating vector $|\lambda\rangle$.

\subsubsection{One-dimensional lattice vertex operator algebras}
Fix an even positive integer $N \in 2\ZZ^{>0}$.  Define
\[V_{\sqrt{N}\ZZ}:=\bigoplus_{m \in \ZZ}\pi_{m\sqrt{N}}. \]

Then $V_{\sqrt{N}\ZZ}$ carries with it a vertex operator algebra structure with $V_0=\pi_0$ as a conformal subalgebra.  We sketch a sketch of a proof of this fact.

A stronger version of the reconstruction theorem says that it is enough to define the fields $Y(|\lambda\rangle,z)=:V_\lambda(z)$ for each $\lambda \in \sqrt{N}\ZZ$.

The calculations for this will involve a pair of numbers $c_{\lambda,\mu} \in \CC$ satisfying the following equations:
\begin{align}
  c_{\lambda,0}&=1 \quad \forall \lambda \in \sqrt{N}\ZZ \label{cocycle1} \\
  c_{\lambda,\mu+\nu}c_{\mu,\nu}&=(-1)^{\lambda\mu}c_{\mu,\lambda+\nu}c_{\lambda,\nu}\quad \forall \lambda,\mu\,nu \in \sqrt{N}\ZZ.\label{cocycle2}
\end{align}
Note that the power of $(-1)$ in (\ref{cocycle2}) is in fact always equal to $1$ and so $c_{\lambda,\mu}=1$ is a valid solution; however, for multidimensional lattices, this is not necessarily true.

Denote by $S_\lambda \in \End V_{\sqrt{N}\ZZ}$ the linear map such that
\begin{align}
  S_\lambda|\mu\rangle &= c_{\lambda,\mu}|\lambda+\mu\rangle\\
  [S_\lambda,b_n] &= 0 \quad \forall n \ne 0.
\end{align}
Then, we can define $V_\lambda(z)$ as follows:
\[
V_\lambda(z)=S_\lambda z^{\lambda b_0}\exp{\left(-\lambda\sum_{n<0}\frac{b_n}{n}z^{-n} \right)} \exp{\left(-\lambda\sum_{n>0}\frac{b_n}{n} z^{-n}\right)}.
\]
Note that the sum over $n>0$ consists of terms that, for every element of $V_{\sqrt{N}\ZZ}$, will be eventually zero, and so the exponentiation makes sense; the $z^{\lambda b_0}$ term makes sense, as for any $\mu \in \sqrt{N}\ZZ$, $\mu\lambda \in \ZZ$.  The condition (\ref{cocycle1}) arises from confirming the vacuum axiom, and the condition (\ref{cocycle2}) arisis from confirming the locality axiom.
\subsubsection{Lattice vertex operator algebra for more general lattices}
Let $L$ be a lattice of finite rank, equipped with a bilinear form
\[(-,-):L \times L \rightarrow \ZZ \]
such that
\[(\lambda,\lambda)\in 2\ZZ^{>0} \quad \forall \lambda \ne 0. \]
Set $\hh=L \otimes_\ZZ \CC$; $\hh$ has an inner product induced by the inner product on $L$.

Define the multidimensional Heisenberg Weyl algebra $\widetilde{H}_L$ as the algebra generated by
\[h_n \quad h\in \hh, n \in \ZZ \]
with the following relations:
\begin{itemize}
\item For fixed $n$, the linear structure of $\hh$ applies to the elements of $h_n$; that is, only a basis for $\hh$ is needed as generators.
\item \[[h_n,g_m]=n(h,g)\delta_{n,-m}\quad \forall g,h \in \hh; m,n \in \ZZ. \]
\end{itemize}
There is always an isomorphism $\widetilde{H}_L \simeq \widetilde{H} \otimes ... \widetilde{H}$, where the number of tensor factors is equal to the rank of $L$.

For each $\lambda \in L$, there is a $\widetilde{H}_L$-module $\pi_\lambda$, generated by $|\lambda\rangle$ with relations
\begin{itemize}
\item $h_n|\lambda\rangle=0 \quad n>0$
\item $h_0|\lambda\rangle=(\lambda,h)|\lambda\rangle,$
\end{itemize}
as one might expect the multi-dimensional generalization to be.  Similarly, $\pi_0$ still carries a vertex algebra structure.

Define
\[V_L=\bigoplus_{\lambda \in L}\pi_\lambda, \]
with the grading such that $\deg(h_n)=-n$ and $\deg(|\lambda\rangle)=\frac{1}{2}(\lambda,\lambda)$.

To define a vertex operator algebra structure on $V_L$, we will need coefficients $c_{\lambda,\nu} \in \CC$ for $\lambda,\nu \in L$ satisfying:
\begin{align}
  c_{\lambda,0}=c_{0,\lambda}&=1 \label{latcocyc1}\\
  c_{\lambda,\mu}&=(-1)^{(\lambda,\mu)}c_{\mu,\lambda} \label{latcocyc2}\\
  c_{\mu,\nu}c_{\mu+\nu,\lambda}&=c_{mu,\nu+\lambda}c_{\nu,\lambda}. \label{latcocyc3}
\end{align}
Assume that $c_{\lambda,\mu} \ne 0 \forall \lambda,\nu \in L$.  Then, (\ref{latcocyc1}) and (\ref{latcocyc3}) imply that the function $L \times L \rightarrow \CC^\times$ given by the coefficients of $c$ is a $2$-cocycle of $L$ with coefficients in $\CC^\times$.  Scaling $|\lambda\rangle$ by $a_\lambda$ yields the scaling
\[c_{\lambda,\mu} \mapsto a_\lambda a_\mu a_{\lambda+\mu}^{-1} c_{\lambda,\mu}, \]
which scales $c$ by a coboundary.

Therefore, solving (\ref{latcocyc1}-\ref{latcocyc3}) is equivalent, up to scaling of the $|\lambda\rangle$s, to finding an element of $H^2(L,\CC^\times)$ satisfying (\ref{latcocyc2}).  As $L$ is an abelian group, we know $H^2(L,A) \simeq \Lambda^2L^* \otimes A$ for any abelian group $A$, and that this isomorphism is determined precisely by the values forced by (\ref{latcocyc2}).  From this, we can show that not only is it possible to find such a set of values, and that they are unique up to scaling, but that with the correct scaling we can set $c_{\lambda,\mu}=\pm 1 \quad \forall \lambda,\mu$.

The conformal vector of $V_L$ is
\[\omega=\frac{1}{2} \sum_{i=1}^{\dim \hh}(h_i)_{-1}(h^i)_{-1}, \]
where $\{h_i\}$ and $\{h^i\}$ are dual bases of $\hh$ using $(-,-)$.
\subsubsection{Relation between Kac-Moody and lattice vertex algebras}
Let $L$ be a lattice of type $ADE$ and $\gf$ be the corresponding simple Lie algebra.  Let $\ghat$ be the corresponding affine Lie algebra to $\gf$.  Consider the vertex operator algebra $V_L$.

The multidimentional Heisenberg Weyl algebra $\widetilde{H}_L$ acts on $V_L$, and we also have fields corresponding to the lattice elements; for $\lambda \in L$, write
\[Y(|\lambda\rangle,z)=\sum_{j \in \ZZ}v^{(j)}_\lambda z^{-j-1}. \]

Then, there is a morphism $\sigma: \ghat \rightarrow \End(V_L)$ given by:
\begin{align}
  K &mapsto 1\\
  u \otimes t^n &\mapsto u_n \quad u \in \hh, n \in \ZZ\\
  E_\alpha \otimes t^n &\mapsto v^{(n)}_\alpha \quad \alpha \in \Delta, n \in \ZZ.
\end{align}
The vertex operator algebra $V_L$ will be a quotient of the vertex operator algebra $V_1(\gf)$, and in fact will be the simple quotient $L_1(\gf)$.

Conversely, it is possible to extract a definition for $\gf$ from $V_L$: note that
\[(V_L)_1 = \hh \vac \oplus \bigoplus_{(\lambda,\lambda)=2}|\lambda\rangle, \]
and that locality and associativity will imply the Jacobi identity for the bracket
\[[g,h]=g_{(0)}h \quad g,h \in V_1 \]
for any vertex operator algebra $V$.
\end{document}
 